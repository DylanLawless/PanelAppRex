\input{head.tex}
\usepackage[printonlyused,withpage,nohyperlinks]{acronym}

\begin{document}
\title{PanelAppRex aggregates disease gene panels and facilitates sophisticated search}	
\author[1]{Dylan Lawless\thanks{Addresses for correspondence: \href{mailto:Dylan.Lawless@kispi.uzh.ch}{Dylan.Lawless@kispi.uzh.ch}}}

\affil[1]{Department of Intensive Care and Neonatology, University Children's Hospital Zürich, University of Zürich, Switzerland.}

\maketitle
\justify

\maketitle
Word count: 1342

\begin{abstract}
\noindent
\textbf{Motivation:} Gene panel data provides critical insights into disease-gene correlations. However, aggregating and interrogating this diverse dataset can be challenging. PanelAppRex addresses this by first preparing a machine-readable aggregate and second by offering a sophisticated natural search interface that streamlines data exploration for both clinical and research applications.\\[1ex]
\textbf{Results:} PanelAppRex aggregates gene panel data from source including CliVar, UniProt, and \ac{ge}’s PanelApp, including the approved panels used in the NHS National Genomic Test Directory and the 100,000 Genomes Project. It enables users to execute complex queries by gene names, phenotypes, disease groups and more, returning integrated datasets in multiple downloadable formats. Benchmarking demonstrates 
93\% - 100\% accuracy, effectively simplifying variant discovery and interpretation to enhance workflow efficiency. The greatest benefit is the analysis ready format for bioinformatic integration. \\[1ex]
\noindent \textbf{Availability:} \url{https://switzerlandomics.ch/services/panelAppRexAi/} (A standalone webpage will be substituted for publication version).
The source code and data are accessible at \url{https://github.com/DylanLawless/PanelAppRex}. PanelAppRex is available under the MIT licence. 
The dataset is maintained for a minimum of two years following publication.
\end{abstract}
\clearpage


\section*{Acronyms}
\renewenvironment{description} % Internally acronym uses description which we redefine to make condense spacing. 
{\list{}{\labelwidth0pt\itemindent-\leftmargin
    \parsep-1em\itemsep0pt\let\makelabel\descriptionlabel}}
               {\endlist}
\begin{acronym} 
\acro{rag}[RAG]{Retrieval-augmented generation}
\acro{api}[API]{Application Programming Interface}
\acro{csv}[CSV]{comma-separated values}
\acro{ge}[GE]{Genomics England}
\acro{html}[HTML]{HyperText Markup Language}
\acro{tsv}[TSV]{Tab-separated values}
\acro{rds}[Rds]{R Data Serialization format}
\acro{pid}[PID]{Primary immunodeficiency}
\acro{nhs}[NHS]{National Health Service}
\acro{mit}[MIT]{Massachusetts Institute of Technology}
\acro{jaci}[JACI]{Journal of Allergy and Clinical Immunology}
\acro{pdf}[PDF]{Portable Document Format}

 \end{acronym} 
 
 
\section{Introduction}
\noindent
Gene panels are pivotal for the diagnosis and interpretation of genetic disorders. Sources like \ac{ge}’s PanelApp and PanelApp Australia host comprehensive panels that support genomic testing \cite{martin_panelapp_2019}. 
For instance, these are integral in the NHS and research projects such as the 100,000 Genomes Project\cite{martin_panelapp_2019}. Despite its utility, manual panel selection and data aggregation remain labour intensive. PanelAppRex was developed to simplify this process by automating data retrieval from \ac{api} sources and integrating the data into machine and user-friendly formats. We include the use of \ac{ge}’s PanelApp, ClinVar, and UniProt
\cite{martin_panelapp_2019,
landrum_clinvar_2018, the_uniprot_consortium_uniprot_2025}. 
The novel natural language-style search capability further streamlines the discovery of disease gene panels by allowing queries based on gene names, phenotypes, disease groups and additional key attributes which were supplemented with evidence-based \ac{rag}.

\section{Materials and methods}
\subsection{Implementation}
\noindent
PanelAppRex was implemented in R and integrates data from \ac{ge}’s PanelApp, ClinVar, and UniProt
\cite{martin_panelapp_2019,
landrum_clinvar_2018, the_uniprot_consortium_uniprot_2025}. 
It performed credentialed access to the API to retrieve all approved panels, merging them into two formats: a simplified version (Panel ID, Gene) and a complex version (including metadata such as confidence level, mode of inheritance, and disease information), and several metadata summary statistics. In addition, the tool incorporates a search module to execute complex user queries. The search functionality supports queries by gene names, phenotypes, disease names, disease groups, panel names, genomic locations and other identifiers. 
RAG was used to improve the natural queries in hidden states based on evidence about the disease and gene function by supplementing the data with additional sources including ClinVar, UniProt, etc. 

\subsection{Usage}
\noindent
Online, queries can be executed via the integrated search bar in our \ac{html} version, where a JavaScript function splits the query into individual terms and progressively filters rows - retaining only those that match all active terms while ignoring unmatched ones. This enables users to perform complex, partial matching queries (e.g. ``paediatric RAG1 primary immunodeficiency skin disorder'') to rapidly identify the panel most closely associated with their hypothesis on primary immunodeficiency and paediatric skin disorders.

Bioinformatically, users can import the provided, ready-for-use, datasets in TSV or Rds formats. Users are most likely to merge with their own omic data based on merging with gene/protein ID.
The following code snippet, available in \texttt{minimal\_example.R}, demonstrates how to load the data in R:
\begin{verbatim}
# TSV format
path_data <- "../data"
core_path <- paste0(path_data, "/PanelAppData_combined_core")
minimal_path <- paste0(path_data, "/PanelAppData_combined_minimal")

df_core <- read.table(
    file= paste0(core_path, ".tsv"), 
    sep = "\t", header = TRUE)

df_minimal <- read.table(
    file= paste0(minimal_path, ".tsv"), 
    sep = "\t", header = TRUE)

# Rds format
rds_path <- paste0(path_data, "/PanelAppData_combined_Rds")
df_core <- readRDS(file= rds_path)
\end{verbatim}


\section{Results}
\noindent
PanelAppRex successfully aggregates data, currently from 451 panels, and several genomics databases to offer a user-friendly search functionality (\textbf{Figure \ref{fig:performance}}).
Users can retrieve results filtered by gene names, phenotypes, disease groups and other criteria. The system returns a table view with panel details and provides options for exporting results in \ac{csv}, Excel, or \ac{pdf} formats.
Bioinformatic uses may include virtual panels, for constructing prior odds, or for formal reporting on qualifying variants protocols.

\begin{figure}[ht]
    \centering
    \includegraphics[width=0.49\textwidth]{screenshot_1.png}
    \includegraphics[width=0.49\textwidth]{screenshot_2.png}    
    \includegraphics[width=0.85\textwidth]{benchmark.pdf}    
 %  \caption{PanelAppRex interface displaying search results for a complex natural language query. The top panels show screenshots: the left displays the full database before filtering, and the right shows detailed results for a selected panel. Below, benchmark metrics are presented: (A) the number of panels queried per case study, (B) the percentage of panels that included the true causal gene, and (C) the percentage from the single best panel selection that contained the true causal gene.}

\caption{PanelAppRex interface displaying search results for a complex query. Top: screenshots showing the search interface, with the left panel displaying the full database before filtering and the right panel showing detailed results for a selected panel. Bottom: benchmark metrics are presented as follows: (A) for each of the 5 case studies, the total number of panels returned, the number of panels that included the causal gene, and the single best panel selected (always 1 by default); (B) the percentage of all returned panels that included the true causal gene; (C) the percentage of the single best panels that contained the true causal gene.}


    \label{fig:performance}
\end{figure}

\subsection{Validation benchmarking}
\noindent
To mimic a clinician diagnosing a new disease such as \ac{pid}, we began by systematically selecting genetic diagnosis case studies from the current online catalogue from the \ac{jaci}, using the first five results  \cite{arruda_genetic_2015, 
mcaleer_severe_2015,
verhoeven_hematopoietic_2022,
magerus-chatinet_autoimmune_2013,
sharfe_fatal_2014}. We chose this source because our research centres on the genetic diagnosis of \ac{pid}.
The clinical background from these studies was used to construct keyword queries from patient features, simulating a naïve starting position for a clinician (full sources, queries, and results in \textbf{supplemental table 1}). We then tested whether our PanelAppRex tool could successfully retrieve panels that included the final causal gene reported in each case study. 

The tool's query process retrieves gene panels using natural language-style input. For example, in the first case study on Hereditary Angioedema, the clinician reported suspecting that the condition was linked to ``\textit{SERPING1}, Factor XII, and edema" which we used as the query terms. Although the true causal gene, \textit{F12}, was not mentioned, the inclusion of the related disease terms enabled the system to correctly retrieve the panel containing \textit{F12} based on the hidden search knowledge.

In our evaluation, PanelAppRex returned panels in which the causal gene was present in 93\% of all returned panels, with the user-selected best panel achieving a perfect accuracy of 100\% 
(\textbf{Figure \ref{fig:performance}}).
 These metrics were derived by comparing the causal gene identified from the case study abstracts to the panels returned by our query, and by applying a subjective relevance measure to mimic intuitive panel selection - acknowledging that certain panels, such as the established PID gene panel, are inherently more reliable than broader, less specific panels. Overall, the results confirm that our approach accurately identifies the most relevant panels and effectively supports clinical decision-making in complex diagnostic scenarios.

\FloatBarrier
\section{Summary}
\noindent
PanelAppRex offers a robust solution for aggregating and querying gene panel data. Its sophisticated search feature simplifies data exploration and enhances variant interpretation. Future work will focus on expanding the range of supported queries and integrating additional genomic data sources, further supporting the needs of clinicians and researchers.

\section*{Acknowledgements}
\noindent
We acknowledge \ac{ge} for providing public access to the PanelApp data.
The use of data from \ac{ge} panelapp was licensed under the Apache License 2.0.
The use of data from UniProt was licensed under Creative Commons Attribution 4.0 International (CC BY 4.0).
ClinVar asks its users who distribute or copy data to provide attribution to them as a data source in publications and websites \cite{landrum_clinvar_2018}.

\section*{Competing interest}
\noindent
We declare no competing interest. 

\bibliographystyle{unsrtnat}
\bibliography{references.bib}
\clearpage

\end{document}